% easychair.tex,v 3.4 2015/12/10

\documentclass{easychair}
%\documentclass[EPiC]{easychair}
%\documentclass[debug]{easychair}
%\documentclass[verbose]{easychair}
%\documentclass[notimes]{easychair}
%\documentclass[withtimes]{easychair}
%\documentclass[a4paper]{easychair}
%\documentclass[letterpaper]{easychair}

\usepackage{doc}

% use this if you have a long article and want to create an index
% \usepackage{makeidx}

% In order to save space or manage large tables or figures in a
% landcape-like text, you can use the rotating and pdflscape
% packages. Uncomment the desired from the below.
%
% \usepackage{rotating}
% \usepackage{pdflscape}

% Some of our commands for this guide.
%
\newcommand{\easychair}{\textsf{easychair}}
\newcommand{\miktex}{MiK{\TeX}}
\newcommand{\texniccenter}{{\TeX}nicCenter}
\newcommand{\makefile}{\texttt{Makefile}}
\newcommand{\latexeditor}{LEd}

%\makeindex

%% Front Matter
%%
% Regular title as in the article class.
%
\title{ Put Your Full Title In Here }

% Authors are joined by \and. Their affiliations are given by \inst, which indexes
% into the list defined using \institute
%

\author{ 
	     
	    Bert Niehaus\inst{1}\thanks{Thanks 1} 
	    
  \and 
	    Paul Meyer\inst{2}\thanks{Thanks 2} 
	    
  \and 
	    Anna Miller\inst{3}\thanks{Thanks 3} 
	    
  \and 
	    Charles Sinner\inst{4}\thanks{Thanks 4}
}

% Institutes for affiliations are also joined by \and,
\institute{
	 
	   University Koblenz-Landau \\
	   \email{niehaus@uni-landau.de}
	
  \and 
	   Uni Meyer \\
	   \email{meyer@mail.com}
	
  \and 
	   Uni Miller \\
	   \email{miller@mail.com}
	
  \and 
	   Uni Meyer \\
	   \email{Siner@mail.com}
}
%  \authorrunning{} has to be set for the shorter version of the authors' names;
% otherwise a warning will be rendered in the running heads. When processed by
% EasyChair, this command is mandatory: a document without \authorrunning
% will be rejected by EasyChair

\authorrunning{Bert Niehaus, Paul Meyer, Anna Miller, Charles Sinner}

% \titlerunning{} has to be set to either the main title or its shorter
% version for the running heads. When processed by
% EasyChair, this command is mandatory: a document without \titlerunning
% will be rejected by EasyChair
\titlerunning{Put Your Short Title In Here}



\begin{document}

\maketitle

\begin{abstract}Put Your Abstract In Here\end{abstract}

% The table of contents below is added for your convenience. Please do not use
% the table of contents if you are preparing your paper for publication in the
% EPiC series

\setcounter{tocdepth}{2}
% {\small \tableofcontents}

%\section{To mention}
%
%Processing in EasyChair - number of pages.
%
%Examples of how EasyChair processes papers. Caveats (replacement of EC
%class, errors).

\pagestyle{empty}

%------------------------------------------------------------------------------
\section{Markdown to Latex Using APA 6
Style}\label{markdown-to-latex-using-apa-6-style}

If you have to use the American Psychological Association Style version
6 (apa6) then this package of templates and scripts might be just what
you need.

Using the style if you have to do it manually in Word is a pain. These
templates and scripts allow you to automatically generate apa6 compliant
documents with the minimum of fuss and effort. Your citations are also
created and formatted properly.

\subsection{The Process}\label{the-process}

These templates and scripts use Markdown as the document markup
language, they then convert the Markdown document to LaTeX and run biber
to format the bibliography and citations and finally convert the whole
document to a properly formatted PDF. That means that you need some
software installed on your system before you can begin.

\subsection{Pre-Requisites}\label{pre-requisites}

The set up is the most time consuming thing about this package. Once you
have set up your system then creating documents is quick and simple. You
need to be able to issue simple commands at the terminal but most people
can manage to do that.

This set up is tested to work on MacOS 10.8.2 using Pandoc 1.11.1 . It
will no doubt work on Linux with perhaps minor mods to the paths in the
scripts. It will also probably work on Windows using a shell emulator. I
don't know about Windows so you are on your own there!

First you need to install
\href{http://johnmacfarlane.net/pandoc/installing.html}{Pandoc} . It's
quick and simple and once installed it's seamless to run. Once you have
installed you should open Terminal and at the prompt type
\texttt{pandoc -{}-version} to check that you have a good install. If
Pandoc tells you its version number you are good to go.

Now you need to install a LaTeX distribution. This is a bigger job - the
distributions are quite large. \href{http://tug.org/mactex/}{Here's} a
good Mac distribution. There are distributions for other platforms.

Make a cup of coffee and set the install running. Once you have LaTeX
installed find the TeX Live Utility and launch it. Use it to install the
\texttt{apa6} package. You will need this package to format your
documents according to APA style.

Once you have all of those things ready you need to install these
scripts.

\subsection{Installation}\label{installation}

Either clone the Git repo off Bitbucket or download the zip file from
the same location. Put the scripts into a directory which will be the
directory you use to generate your documents. We'll call this
\texttt{maindir}.

Take the \texttt{apa6template.tex} document and put it into
\texttt{/usr/local/share/pandoc-1.11/data/templates/apa6template.tex} .
That path will be slightly different if you have a different version of
Pandoc installed. Modify it to suit.

There are now a number of files left in \texttt{maindir}:

\begin{itemize}
\itemsep1pt\parskip0pt\parsep0pt
\item
  \texttt{abstract.md} is where you enter your document abstract
\item
  \texttt{affiliations.md} is where you enter your affiliations for the
  title page
\item
  \texttt{APATemplate.md} is a Markdown template that is set up for you
  to use to actually write your document. Copy and rename it to use if
  for your own document.
\item
  \texttt{shorttitle.md} is where you enter the short title for your
  document
\item
  \texttt{keywords.md} which is where you enter your document keywords.
\item
  \texttt{createpdf.sh} is a script which you will use to create your
  final PDF document. It runs PDFLaTeX and biber multiple times to
  create the document, the citations and the bibliography.
\item
  \texttt{md2pdf.sh} creates a PDF but doesn't run multiple passes or
  run biber. It is for doing a quick document generate to check
  formatting and maths.
\item
  \texttt{md2tex.sh} takes your Markdown document and generates a LaTeX
  document from it. This is useful if you want to debug problems in your
  document.
\item
  \texttt{tex2pdf.sh} is the script that you'd use after you used the
  previous script. For instance if you made mods to the tex file you
  could then convert to PDF using this script.
\item
  \texttt{cleantex.sh} when you run the build command LaTeX creates all
  sorts of additional files in \texttt{maindir}. Sometimes you feel like
  you'd like to clean those files up. Sometimes you find aberrant
  behaviour - particularly when you are debugging a .tex file. This
  script can help be removing much of the detritus from
  \texttt{maindir}. Use with care and have a look at it and comment out
  any file types you don't want removed.
\end{itemize}

You need to have a bibliography file in \texttt{maindir}. The name of
this file is set in \texttt{apa6template.tex}. Currently it is set to
\texttt{bibliography.bib} you can either call your \texttt{.bib} file
that or modify the template to suit the name of your \texttt{.bib} file.
If you need a bibliography app, google for Jabref.

You should open Terminal and cd to \texttt{maindir} then run
\texttt{chmod 755 *.sh} this will ensure that your scripts are
executable.

\subsection{Using the Scripts}\label{using-the-scripts}

Here's the short version:

\begin{enumerate}
\def\labelenumi{\arabic{enumi}.}
\itemsep1pt\parskip0pt\parsep0pt
\item
  Copy the \texttt{APATemplate.md} document and write your document in
  the copy you've made.
\item
  Populate the keywords, affiliations, abstract and shorttitle
  documents.
\item
  When you have your document ready in \texttt{maindir} create a PDF of
  it by running \texttt{./createpdf.sh} from within \texttt{maindir}.
  This will create various files in \texttt{maindir} most of which don't
  matter. The two files you are interested in are the file called
  \texttt{yourfilename.tex} and \texttt{yourfilename.pdf}.
\end{enumerate}

Now the long version.

\subsubsection{Your Markdown Document}\label{your-markdown-document}

The first three lines of the Markdown template are meta data that is
passed into the apa6template.tex document by Pandoc. You need to ensure
that these are the first three lines in the document with no blank space
before or between them.

The content of these lines is:

\begin{enumerate}
\def\labelenumi{\arabic{enumi}.}
\itemsep1pt\parskip0pt\parsep0pt
\item
  The document title
\item
  The Author or Authors names
\item
  The format of the final document. This has three options: \texttt{man}
  is the manuscript version of APA 6 style, \texttt{doc} is a plain
  document format and \texttt{jou} is the two column journal style. Set
  this parameter to suit your current purpose.
\end{enumerate}

You can enter either
\href{http://code.ahren.org/markdown-cheatsheet}{Markdown} format text
or verbatim LaTeX or a mixture of both. The ability to enter verbatim
LaTeX is important, particularly if you need to include formulas and
other maths in your document. Pandoc will pass it through
unchanged\ldots{}with a couple of exceptions.

If you want to set in-line maths then you need to escape the slashes in
the environment like this:
\texttt{\textbackslash{}\textbackslash{}(...\textbackslash{}\textbackslash{})}
the additional \texttt{\textbackslash{}} gets stripped on the way
through. If you want to set display maths then the same applies
\texttt{\textbackslash{}\textbackslash{}{[}...\textbackslash{}\textbackslash{}{]}}.

However if you are just setting up another environment you don't need to
double escape:

\begin{verbatim}
\begin{equation}
...
\end{equation}
\end{verbatim}

Will work just fine. Similarly if you set up some other environment and
then put display or inline maths inside that environment you
\emph{don't} need to double escape it. So:

\begin{verbatim}
\begin{APAenumerate}
\item Some text with maths \(...\)
\end{APAenumerate}
\end{verbatim}

Will work fine too.

Have a good read of the PDF document that is installed with the apa6
package using the TeX Live Utility. You can access it by finding the
apa6 package and then double clicking it and then clicking the PDF. It
tells you all about the capabilities of the package and how to use them.

\subsubsection{Running Head on Title
Page}\label{running-head-on-title-page}

If you \emph{don't} want a running head on the title page then find this
text in the preamble of the \texttt{apa6template.tex} document:

\begin{verbatim}
\fancypagestyle{titlepage}{%
  \lhead{}%
   \rhead{\thepage}%
}
\end{verbatim}

Make sure it is \emph{not} commented.

If you \emph{do} want a running head on the title page of your
manuscript - as per strict APA 6 Style then \emph{comment out} the above
text.

\subsection{Issue resolution \&
Bugfixes}\label{issue-resolution-bugfixes}

\subsubsection{Issue 1}\label{issue-1}

A necessary file \texttt{rgb.tex} was missing from the distribution. It
has now been added. It should be placed in \texttt{maindir}.

\subsubsection{Issue 2}\label{issue-2}

Added \texttt{\textbackslash{}usepackage\{fancyhdr\}} to
\texttt{apa6template.tex}.

\subsubsection{Issue 3}\label{issue-3}

This issue arose as a discussion about the use of Pandoc citations as
opposed to using LaTeX style citations as verbatim LaTeX. One of the
users explored the issue and the findings are contained in
\href{https://bitbucket.org/zuline/md2latex/issue/3/pandoc-citations-not-supported}{this
discussion.}

The bottom line is that the best outcomes will likely be achieved by
using the LaTeX citation approach not the Pandoc citation approach.
%------------------------------------------------------------------------------

\label{sect:bib}
%\bibliographystyle{plain}
%\bibliographystyle{alpha}
%\bibliographystyle{unsrt}
%\bibliographystyle{abbrv}
%\bibliographystyle{apalike}
\bibliographystyle{apa}
%\bibliographystyle{authortitle}
\bibliography{easychair}

%------------------------------------------------------------------------------
%\appendix
%\section{First Appendix}
%\label{sect:appendix-first}
%Text
%\section{Second Appendix}
%\label{sect:appendix-second}
%Text

%------------------------------------------------------------------------------
% Index
%\printindex

%------------------------------------------------------------------------------
\end{document}
